\documentclass[10pt, a4paper]{article}

\usepackage{amsmath}

\title{A comparison of algorithms user in traffic control systems}
\author{Daniel Gustafsson}
\date{December 2021}

\begin{document}
\maketitle

\section{Summary}
The authors have studied and compared the efficiency of different traffic control systems in intersections.
Three different systems were studied and compared. Firstly a \textit{pretimed} system where the traffic lights
in the intersection shift based on a fixed timer. Secondly a \textit{deterministic} system where the traffic lights
are shifted based on sensor data. And thirdly a \textit{machine learning} system which uses \textit{Reinforcement Learning}
to try and minimize the average squared waiting time of the traffic.

The systems are compared in a single four-way interchange with varied traffic. The scenarios where simulated using SUMO
(Simulation of Urban MObility) which is a road traffic simulation program. All systems where compared on the ability to
minimize the average squared waiting time of traffic in the intersection. The pretimed system was tested using three different
timers, 20, 30 and 40 seconds. A total of 50 different test cases was generated and used to test the different systems.
The tests where split into four categories of traffic demand, 5\%, 10\%, 15\% and 20\%.

The results show that the deterministic algorithm performed best in all tests followed by the pretimed system with the 20 second
timer. The machine learning algorithm performed similarly to the pretimed system with the 20 and 30 second timers in all tests
except the highest demand test (20\%).

The results may not be tied to reality as real sensors may not perform as simulated which could mean significantly lower results
for the deterministic system. There are also a number of different ways to implement the machine learning algorithm which could
lead to different results for that system.

In conclusion, the deterministic system is the highest performing in all scenarios assuming perfect sensors. Further, the
machine learning algorithm does not perform well in high traffic demand situations. 

I think the results of this thesis is interesting to anyone who is regularly driving in city traffic. At some intersections it feels
like you wait forever and it feels nice that some people are trying to solve that problem. Unfortunately I don't think this thesis
has a solution to traffic problems but at the very least it provides data for further researching.

\section{Scientific considerations}
In this section I will analyze the thesis with different course subjects in mind.
\subsection{Conclusions}
The tests conducted in the thesis are in my opinion too narrow to draw the conclusions that the authors have drawn. In the first conclusion
they simply state that the deterministic traffic control system is best assuming perfect sensors.
The tests in the thesis were all performed using the same interchange layout but the drawn conclusion is in the scope of all interchanges.
Similarly I think that the second conclusion that machine learning algorithms have lower performance than both deterministic and pretimed
systems in high demand scenarios is also faulty. The authors only tested one specific machine learning algorithm implementation but still
draw conclusions about all machine learning algorithms.

\subsection{Reproduction}
To reproduce the study and get the same or similar results I would need the data used to train the machine learning model.
Without this data, the results could vary significantly. Apart from this I think the study is well described and easily reproducible.
The fact that the authors have published all code and configurations used on Github makes the reproduction significantly easier.

\subsection{Ethics}
I find the lack of consideration for pedestrians or cyclist to be a ethical problem with this study. I realize that the addition of pedestrians
and cyclist would make the study magnitudes more complex but to not consider them would be to marginalize a huge group of people.
Many cities in the modern society tries to minimize the number of cars on the roads and encourage alternative ways of transportation
like cycling, walking and public transport. To not take these factors into consideration and instead try to optimize intersections based on
waiting times for cars is outdated and ethically questionable.

\subsection{Arguments}
The authors argue in their method chapter that the three different timers in the pretimed algorithm where chosen to resemble reality and to
be similar to related work. In my own experience, I have observed many traffic lights (in reality) switch state in significantly less than
20 seconds. I also think that since the results show a linear relationship between the different timers where lower timer equals lower average
waiting times, there should have been tests to find where this relationship ends and the bottom is reached. I believe the authors might have
come to different conclusions if the pretimed and learning systems were tested with lower minimum "greentime" (the time the traffic light is green).
The deterministic system didn't have a minimum greentime like the other systems, maybe the results are just a product of that variable.



\section{Suggestions}


\bibliographystyle{amsplain}
\bibliography{essay.bib}

\end{document}