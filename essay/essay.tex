\documentclass[10pt, a4paper]{article}

\usepackage{amsmath}
\usepackage{url}

\title{A comparison of algorithms used in traffic control systems}
\author{Daniel Gustafsson}
\date{December 2021}

\begin{document}
\maketitle

\section{Summary}
The authors have studied and compared the efficiency of different traffic control systems in intersections.
Three different systems were studied and compared. Firstly a \textit{pretimed} system where the traffic lights
in the intersection shift based on a fixed timer. Secondly a \textit{deterministic} system where the traffic lights
are shifted based on sensor data. And thirdly a \textit{machine learning} system which uses \textit{Reinforcement Learning}
to try and minimize the average squared waiting time of the traffic.

The systems are compared in a single four-way interchange with varied traffic. The scenarios where simulated using SUMO
(Simulation of Urban MObility) which is a road traffic simulation program. All systems where compared on the ability to
minimize the average squared waiting time of traffic in the intersection. The pretimed system was tested using three different
timers, 20, 30 and 40 seconds. A total of 50 different test cases was generated and used to test the different systems.
The tests where split into four categories of traffic demand, 5\%, 10\%, 15\% and 20\%.

The results show that the deterministic algorithm performed best in all tests followed by the pretimed system with the 20 second
timer. The machine learning algorithm performed similarly to the pretimed system with the 20 and 30 second timers in all tests
except the highest demand test (20\%).

The results may not be tied to reality as real sensors may not perform as simulated which could mean significantly lower results
for the deterministic system. There are also a number of different ways to implement the machine learning algorithm which could
lead to different results for that system.

In conclusion, the deterministic system is the highest performing in all scenarios assuming perfect sensors. Further, the
machine learning algorithm does not perform well in high traffic demand situations. 

I think the results of this thesis is interesting to anyone who is regularly driving in city traffic. At some intersections it feels
like you wait forever and it feels nice that some people are trying to solve that problem. Unfortunately I don't think this thesis
has a solution to traffic problems but at the very least it provides data for further researching.

The thesis in question can be found here\cite{thesis}.

\section{Scientific considerations}
In this section I will analyze the thesis with different course subjects in mind.

\subsection{Conclusions}
The tests conducted in the thesis are in my opinion too narrow to draw the conclusions that the authors have drawn. In the first conclusion
they simply state that the deterministic traffic control system is best assuming perfect sensors.
The tests in the thesis were all performed using the same interchange layout but the drawn conclusion is in the scope of all interchanges.
Similarly I think that the second conclusion that machine learning algorithms have lower performance than both deterministic and pretimed
systems in high demand scenarios is also faulty. The authors only tested one specific machine learning algorithm implementation but still
draw conclusions about all machine learning algorithms.

\subsection{Reproduction}
To reproduce the study and get the same or similar results I would need the data used to train the machine learning model.
Without this data, the results could vary significantly. Apart from this I think the study is well described and easily reproducible.
The fact that the authors have published all code and configurations used on Github makes the reproduction significantly easier.

\subsection{Ethics}
I find the lack of consideration for pedestrians or cyclist to be a ethical problem with this study. I realize that the addition of pedestrians
and cyclist would make the study magnitudes more complex but to not consider them would be to marginalize a huge group of people.
Many cities in the modern society tries to minimize the number of cars on the roads and encourage alternative ways of transportation
like cycling, walking and public transport. To not take these factors into consideration and instead try to optimize intersections based on
waiting times for cars is outdated and ethically questionable.

Another interesting ethical dilemma regarding this thesis is the use of average \textit{squared} waiting times instead of just average waiting times.
This is used to penalize cases where a few cars have to wait for a long time for the benefit of the many cars.
From a utilitarians point of view this is not obviously correct, why should the many wait for the few? Would it not be better to always maximize the
number of cars that has a green light? The authors seem to take a more Deontology-like approach where they consider it a duty of the traffic
control system to let all cars pass in a reasonable time. 

\subsection{Arguments}
The authors argue in their method chapter that the three different timers in the pretimed algorithm where chosen to resemble reality and to
be similar to related work. In my own experience, I have observed many traffic lights (in reality) switch state in significantly less than
20 seconds. I also think that since the results show a linear relationship between the different timers where lower timer equals lower average
waiting times, there should have been tests to find where this relationship ends and the bottom is reached.
Similarly, the machine learning system is forced to stay in a state without switching for at least 20 seconds. This is something they could have
explored further and I think it would have been interesting to see how the machine learning system would have performed without restraints.

I believe the authors might have come to different conclusions if the pretimed and learning systems were tested with lower minimum "greentime"
(the time the traffic light is green). The deterministic system didn't have a minimum greentime like the other systems, maybe the results are 
just a product of that variable.

\section{Suggestions}
In this section I will make suggestions on how the authors could improve their work to qualify for a Masters' thesis.
I will consider KTH's official goals and criteria for a Masters' thesis\cite{criteria} in all suggestions.
Additionally I will present some new research in the area and explain how it is relevant to this work.

\subsection{Qualifying for a Masters' thesis}
I would argue that the conclusions the authors have drawn are too general to be drawn from the results of their study. My suggestion is to
preferably conduct more experiments in other intersection layouts and other traffic conditions or possibly to narrow down the conclusions
to better fit the results. The conclusions as they are would in my opinion be in violation of the third goal/criteria for a Masters' thesis.

Another suggestion I have is to conduct some additional experiment on how the greentime effects the average squared waiting time. Since the
pretimed system was tested with three different greentime configurations and the lower ones got the better results in every test, I think it 
would be highly relevant to continue lowering the greentime til a limit is reached. This limit should then be the benchmark for the other systems
to compare to. I also suggest completely removing the minimum greentime from the machine learning system since the point of machine learning is
to let the algorithm decide its own inner parameters based on observations. To set a minimum for one of these parameters would be to restrain
the algorithm. This in combination with the fact that the deterministic system didn't have a minimum greentime makes the results of this study
insignificant. Alternatively, a minimum greentime of 20 seconds could be configured for the deterministic system to make the study fair
between the different systems. The choice to have different minimum greentime variables for the different systems is in my opinion a violation
of the third goal/criteria for a Masters' thesis.

The lack of consideration of pedestrians and cyclist is in my opinion in violation to the seventh goal/criteria for a Masters' thesis. I suggest
the authors either write a paragraph explaining the lack of these considerations or that the goal of the thesis be changed to accommodate for
pedestrians and cyclist. To change the goal would also mean that the average squared waiting time for the cars wouldn't be sufficient for comparison
any longer. Safety of unprotected road users needs to be considered and probably prioritized and the minimum greentime needs to be calculated based
on how long it would take a pedestrian to cross the road safely. This would probably add significant complexity to the work and stray from the original
purpose, therefore my suggestion is to include a paragraph to explain that the thesis exclusively considers cars and should not be used as reference
in road planning other scenarios.

\subsection{New research}


\bibliographystyle{amsplain}
\bibliography{essay.bib}

\end{document}