\documentclass[10pt, a4paper]{article}

\usepackage{amsmath}

\title{A comparison of algorithms user in traffic control systems}
\author{Daniel Gustafsson}
\date{December 2021}

\begin{document}
\maketitle

\section{Summary}
The authors have studied and compared the efficiency of different traffic control systems in intersections.
Three different systems were studied and compared. Firstly a \textit{pretimed} system where the traffic lights
in the intersection shift based on a fixed timer. Secondly a \textit{deterministic} system where the traffic lights
are shifted based on sensor data. And thirdly a \textit{machine learning} system which uses \textit{Reinforcement Learning}
to try and minimize the average squared waiting time of the traffic.

The systems are compared in a single four-way interchange with varied traffic. The scenarios where simulated using SUMO
(Simulation of Urban MObility) which is a road traffic simulation program. All systems where compared on the ability to
minimize the average squared waiting time of traffic in the intersection. The pretimed system was tested using three different
timers, 20, 30 and 40 seconds. A total of 50 different test cases was generated and used to test the different systems.
The tests where split into four categories of traffic demand, 5\%, 10\%, 15\% and 20\%.

The results show that the deterministic algorithm performed best in all tests followed by the pretimed system with the 20 second
timer. The machine learning algorithm performed similarly to the pretimed system with the 20 and 30 second timers in all tests
except the highest demand test (20\%).

The results may not be tied to reality as real sensors may not perform as simulated which could mean significantly lower results
for the deterministic system. There are also a number of different ways to implement the machine learning algorithm which could
lead to different results for that system.

In conclusion, the deterministic system is the highest performing in all scenarios assuming perfect sensors. Further, the
machine learning algorithm does not perform well in high traffic demand situations.

\section{Scientific considerations}


\section{Suggestions}


\bibliographystyle{amsplain}
\bibliography{essay.bib}

\end{document}