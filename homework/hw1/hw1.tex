\documentclass[11pt,a4paper]{article}

\usepackage{epigraph}

\begin{document}
\title{Homework 1}
\author{Daniel Gustafsson}
\date{September 2021}
\maketitle
	
\section{Science as Falsification}
\epigraph{``Every good scientific theory is a prohibition: it forbids certain things to happen. The more a theory forbids, the better it is.''}{--- \textup{Karl R. Popper}, Science as Falsification}

Karl Popper states in his essay \textit{Science as Falsification} that a scientific theory that is irrefutable is not really scientific \cite{saf}.
He explicitly states that what some may consider the strongest argument in favor of Freud's and Adlers's respective theories on human psychology,
that they always fit and are confirmed by all thinkable human behavior, is actually their weakness.

This statement supports the 3rd conclusion in his essay. 
Stating that a irrefutable theory is weak because of the fact is at the same as stating that a refutable theory is strong.
From there it is not a far stretch to say that the more refutable or the more a theory forbids, the better it is.

An interesting theory developed by Johann Friedrich Meckel in 1811 was that as humans develop from embryos, we go through a ``fish stage'' on the
path toward biological perfection \cite{fish}. Meckel predicted that embryos would have slits in their necks, like gills, at some stage in their development.
In 1827 slits were found on the necks of embryos which looked like evidence in favor of Meckels theory.
Later in the century Charles Darwin came along and proved that Meckels theory was incorrect, the neck slits on embryos was instead explained by 
shared DNA and a common ancestor between fish and humans. 

\section{What is truth?}
\begin{enumerate}
	\item \textit{The program statement while (true) {} gives an infinite loop}

	I believe this statement to be a correspondence truth because a \textit{while (true) {}} statement will result in an theoretically infinite loop, 
	although because infinity is impossible to measure this can not be proven. It is also possible to exit the loop manually by killing the program or 
	flipping the power switch of the computer running the program. All things considered I still believe this to be a correspondence truth.

	\item \textit{Mergesort has complexity O(n log n)}

	This statement is also true but a coherence truth. This is if I can assume complexity means time complexity, if instead complexity means space complexity 
	then the statement is false. I believe it to be a coherence truth because the time complexity of the algorithm in the statement is dependent on the 
	time complexity of the different steps of algorithm. If merging two arrays, which has a time complexity of O(n), instead had a time complexity of O(1) 
	the mergesort algorithm would instead have time complexity O(log n).

	\item \textit{Apple suffered losses in the consumer market last year}
	
	I believe this statement to be false. I don't know if it true or false but until I have looked it up I would consider it a false statement because it 
	contradicts what I believe.

	\item \textit{Comments make it easier to modify programs}
	
	I would consider this statement true. It doesn't always make it easier to modify a program because of comments, but if this statement is considered true, 
	all programmers would benefit from it. It is therefore a pragmatic truth.

	\item \textit{Agile development provides greater job satisfaction}

	Once again I believe this statement to be false because it contradicts my own belief. I don't know for sure but until I have looked up the facts I consider
	this false.

	\item \textit{Two doses of the mRNA Covid-19 vaccine BNT162b2 give a 95\% protection for adults}

	This is a harder statement because I don't know if it is 100\% correct. I consider this statement to be both an intuitive and pragmatic truth. 
	Intuitive because I have a strong internal conviction that this is true and pragmatic because I believe everyone would benefit from believing this.
	One could also consider the possibility that by \textit{protection} the statement doesn't just mean against the Covid-19 virus but against everything,
	in that case I am convinced the statement is false.

	\item \textit{P is a strict subset of NP}

	I know this statement to be unproven and therefore neither true or false. Most programmers would probably benefit from considering this truth, but if 
	everyone considered it truth then no one would try to prove it wrong.

	\item \textit{This statement is true}

	This statement is both true and false. I think it is a correspondence truth because the statement is true if the statement is true, but I also believe that
	the statement is false if the statement is false. This is a unique statement which is both true and false depending on ones believes. I might even consider 
	this statement to be an intuitive truth because it is so closely related to personal belief.

	\item \textit{This statement is false}

	The opposite of the previous statement, this statement can't be either true nor false. If the statement is true, then it is by definition false and the same
	applies if the statement is false, then it is by definition true. What can be said about this statement is that whether you believe this statement true or 
	false you are wrong.

\end{enumerate}
\begin{thebibliography}{9}

\bibitem{saf}
	Karl R. Popper,
	Science as Falsification,
	Conjectures and Refutations,
	1963.

\bibitem{fish}
	O'Connell, Lindsey, "The Meckel-Serres Conception of Recapitulation". 
	Embryo Project Encyclopedia (2013-07-10).
	ISSN: 1940-5030 http://embryo.asu.edu/handle/10776/5916.

\end{thebibliography}
\end{document}