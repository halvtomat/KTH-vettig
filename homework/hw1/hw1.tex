\documentclass[11pt,a4paper]{article}

\usepackage{epigraph}

\begin{document}
\title{Homework 1}
\author{Daniel Gustafsson}
\date{September 2021}
\maketitle
	
\section{Science as Falsification}
\epigraph{``Every good scientific theory is a prohibition: it forbids certain things to happen. The more a theory forbids, the better it is.''}{--- \textup{Karl R. Popper}, Science as Falsification}

Karl Popper states in his essay \textit{Science as Falsification} that a scientific theory that is irrefutable is not really scientific \cite{saf}.
He explicitly states that what some may consider the strongest argument in favor of Freud's and Adlers's respective theories on human psychology,
that they always fit and are confirmed by all thinkable human behavior, is actually their weakness.

This statement supports the 3rd conclusion in his essay. 
Stating that a irrefutable theory is weak because of the fact is at the same as stating that a refutable theory is strong.
From there it is not a far stretch to say that the more refutable or the more a theory forbids, the better it is.

An interesting theory developed by Johann Friedrich Meckel in 1811 was that as humans develop from embryos, we go through a ``fish stage'' on the
path toward biological perfection \cite{fish}. Meckel predicted that embryos would have slits in their necks, like gills, at some stage in their development.
In 1827 slits were found on the necks of embryos which looked like evidence in favor of Meckels theory.
Later in the century Charles Darwin came along and proved that Meckels theory was incorrect, the neck slits on embryos was instead explained by 
shared DNA and a common ancestor between fish and humans. 

\section{What is truth?}
Tjena tjena

\begin{thebibliography}{9}

\bibitem{saf}
	Karl R. Popper,
	Science as Falsification,
	Conjectures and Refutations,
	1963.

\bibitem{fish}
	O'Connell, Lindsey, "The Meckel-Serres Conception of Recapitulation". 
	Embryo Project Encyclopedia (2013-07-10).
	ISSN: 1940-5030 http://embryo.asu.edu/handle/10776/5916.

\end{thebibliography}
\end{document}