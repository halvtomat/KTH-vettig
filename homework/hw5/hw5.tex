\documentclass[11pt, a4paper]{article}

\usepackage{amsmath}

\title{Reading a scientific article}
\author{Daniel Gustafsson}
\date{November 2021}

\begin{document}
\maketitle

\section{Reading an article}
I choose to read the following article:

Kevin S. Killourhy and R. Maxion (2009). Comparing anomaly-detection algorithms for keystroke dynamics.

\subsection{Where was the article published?}

The article was originally published in a conference proceeding by IEEE.
A conference proceeding is a collection of published research papers related to a specific conference or workshop.
The conference in question was held by IEEE which is a great computer science, electric and electronic engineering institution.
IEEE is recognized as the greatest producer of literature in its respective fields.
All IEEE publications are peer reviewed according to their peer review process.

I think the authors choose to publish with IEEE because of IEEE's recognition in the scientific community.

\subsection{Fill in the template}
\textit{Complete citation}

K. S. Killourhy and R. A. Maxion, 
"Comparing anomaly-detection algorithms for keystroke dynamics,"
2009 IEEE/IFIP International Conference on Dependable Systems \& Networks,
2009, pp. 125-134,
doi: 10.1109/DSN.2009.5270346.

\vspace{2mm}\noindent\textit{If web access}

https://ieeexplore.ieee.org/document/5270346 , 3 Nov - 2021

\vspace{2mm}\noindent\textit{Key Words}

Heuristic algorithms,
Detectors,
Rhythm,
Error analysis,
Laboratories,
Computer science,
Algorithm design and analysis,
Benchmark testing,
Security,
Biometrics

\vspace{2mm}\noindent\textit{General subject}

Evaluating anomaly-detection in keystroke dynamics as a tool in computer security.  

\vspace{2mm}\noindent\textit{Specific subject}

Comparing the different existing anomaly-detection algorithms to see which ones have the lowest error rate
and are worth further exploring. Providing a benchmark for future research in the area.

\vspace{2mm}\noindent\textit{Hypothesis}

There exists a procedure and data set which allows anomaly-detection algorithms in keystroke dynamics to be
compared and fairly evaluated against each other. 

\vspace{2mm}\noindent\textit{Methodology}

Create data set from subjects repeatedly typing password.
Implement detector algorithms.
Evaluate algorithms on data set according to procedure. 

\vspace{2mm}\noindent\textit{Results}

A clear division between the best and worst 7 of the 14 algorithms in both performance measures suggesting
the worst 7 are not competitive in this evaluation.
The Nearest Neighbor (Mahalanobis) detector is sole top performer in both measures. 

\vspace{2mm}\noindent\textit{Summary of key points}

A data set and evaluation procedure is established, boosting future research.
It was established which detectors have the lowest error rates on their data set (Nearest Neighbor (Mahalanobis)).

\vspace{2mm}\noindent\textit{Context}

This article is the first to provide an open data set and evaluation procedure for
anomaly-detection algorithms. 

\vspace{2mm}\noindent\textit{Significance}

This work is significant in providing a baseline from where further research can be conducted.

\vspace{2mm}\noindent\textit{Important Figures and/or Tables}

Table 2. Page 8 in article, Page 132 in conference proceeding.

The table displays the evaluation results of the different detector algorithms in performance order.

\vspace{2mm}\noindent\textit{Cited References to follow up on}

Reference 6, 
G. Forsen, M. Nelson, and R. Staron,
Jr. Personal attributes authentication techniques. Technical Report RADC-TR-77-333,
Rome Air Development Center,
October 1977.

\subsection{Referenced article}

Reference 6, 
G. Forsen, M. Nelson, and R. Staron,
Jr. Personal attributes authentication techniques. Technical Report RADC-TR-77-333,
Rome Air Development Center,
October 1977.

This referenced article is interesting because it is the first article to describe the idea of detecting
intruders in a system based on keystroke dynamics.

\subsection{Referencing article}

N. D'Lima and J. Mittal,
"Password authentication using Keystroke Biometrics," 2015 International Conference on Communication,
Information \& Computing Technology (ICCICT),
2015,
pp. 1-6, doi: 10.1109/ICCICT.2015.7045681.

This referencing article is interesting because it explores the possibility to use keystroke dynamics as
a unique identifier of a user and implement authentication based on it.

\subsection{IMRD Format discussion}

In the article:

Kevin S. Killourhy and R. Maxion (2009). Comparing anomaly-detection algorithms for keystroke dynamics.

I would argue that the IMRD format is followed. The first 2 chapters are introduction-related,
the next 4 are methodology related, next one is the results and the last 2 is discussion.

\vspace{4mm}

\noindent In the article:

Noah Snavely, Steven M. Seitz, Richard Szeliski (2006). Photo tourism: Exploring photo collections in 3D.

It seems to me that the structure is similar to IMRD but it is not as certain. The article begins with some
introduction-related chapters and ends with results and discussions but where the methodology usually goes,
there are chapters describing the researched system. 

\vspace{4mm}

\noindent Based on my limited knowledge I think most computer science related papers could be written in
the IMRD format. As far as I know, there are no sub fields where the format isn't appropriate.  

\subsection{How many citations?}

My chosen article has been cited 134 times in IEEE published papers and 90 times in other papers.

\subsection{Essence of contribution}

Providing a performance benchmark for anomaly-detection algorithms in keystroke dynamics.

\end{document}