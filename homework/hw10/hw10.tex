\documentclass[11pt, a4paper]{article}

\usepackage{amsmath}
\usepackage{graphicx}

\title{Statistics and scientific reproducibility}
\author{Daniel Gustafsson}
\date{December 2021}

\begin{document}
\maketitle

\section{Confidence intervals and statistical significance}

I would probably use a confidence interval to present the accuracy 
of the result. A small confidence interval means that the my results
are statistically significant and that they can be trusted. 

I could also use a box plot to visualize my findings (like in figure \ref{fig:plot1}).
In figure \ref{fig:plot1} it is quite obvious that the measured values mostly agree
with each other and that the median is $\sim 10$ 


\begin{figure}[h]
	\centering
	\caption{Box plot of experiment results}
	\input{plot1}
	\label{fig:plot1}
\end{figure}

\vspace{4mm}\noindent
\textit{Second data set is introduced}

\vspace{2mm}\noindent
Figure \ref{fig:plot2} presents the difference in the data sets. Just from looking at the
figure I would claim the results are consistent with each other.

\[\text{Mean original} = 9.8689, \text{ Mean new} = 9.63796\]
\[\text{Median original} = 9.921014, \text{ Median new} = 9.757105\]
\[\text{Variance original} = 2.0353152, \text{ Variance new} = 3.799425\]

From these calculated numbers it is obvious that the new experiment had more varying results.
To fully understand if the results have a statistically significant difference I would have to
perform some test (t-test).

\begin{figure}[h]
	\centering
	\caption{Box plots of the different data sets}
	% GNUPLOT: LaTeX picture with Postscript
\begingroup
  \makeatletter
  \providecommand\color[2][]{%
    \GenericError{(gnuplot) \space\space\space\@spaces}{%
      Package color not loaded in conjunction with
      terminal option `colourtext'%
    }{See the gnuplot documentation for explanation.%
    }{Either use 'blacktext' in gnuplot or load the package
      color.sty in LaTeX.}%
    \renewcommand\color[2][]{}%
  }%
  \providecommand\includegraphics[2][]{%
    \GenericError{(gnuplot) \space\space\space\@spaces}{%
      Package graphicx or graphics not loaded%
    }{See the gnuplot documentation for explanation.%
    }{The gnuplot epslatex terminal needs graphicx.sty or graphics.sty.}%
    \renewcommand\includegraphics[2][]{}%
  }%
  \providecommand\rotatebox[2]{#2}%
  \@ifundefined{ifGPcolor}{%
    \newif\ifGPcolor
    \GPcolorfalse
  }{}%
  \@ifundefined{ifGPblacktext}{%
    \newif\ifGPblacktext
    \GPblacktexttrue
  }{}%
  % define a \g@addto@macro without @ in the name:
  \let\gplgaddtomacro\g@addto@macro
  % define empty templates for all commands taking text:
  \gdef\gplbacktext{}%
  \gdef\gplfronttext{}%
  \makeatother
  \ifGPblacktext
    % no textcolor at all
    \def\colorrgb#1{}%
    \def\colorgray#1{}%
  \else
    % gray or color?
    \ifGPcolor
      \def\colorrgb#1{\color[rgb]{#1}}%
      \def\colorgray#1{\color[gray]{#1}}%
      \expandafter\def\csname LTw\endcsname{\color{white}}%
      \expandafter\def\csname LTb\endcsname{\color{black}}%
      \expandafter\def\csname LTa\endcsname{\color{black}}%
      \expandafter\def\csname LT0\endcsname{\color[rgb]{1,0,0}}%
      \expandafter\def\csname LT1\endcsname{\color[rgb]{0,1,0}}%
      \expandafter\def\csname LT2\endcsname{\color[rgb]{0,0,1}}%
      \expandafter\def\csname LT3\endcsname{\color[rgb]{1,0,1}}%
      \expandafter\def\csname LT4\endcsname{\color[rgb]{0,1,1}}%
      \expandafter\def\csname LT5\endcsname{\color[rgb]{1,1,0}}%
      \expandafter\def\csname LT6\endcsname{\color[rgb]{0,0,0}}%
      \expandafter\def\csname LT7\endcsname{\color[rgb]{1,0.3,0}}%
      \expandafter\def\csname LT8\endcsname{\color[rgb]{0.5,0.5,0.5}}%
    \else
      % gray
      \def\colorrgb#1{\color{black}}%
      \def\colorgray#1{\color[gray]{#1}}%
      \expandafter\def\csname LTw\endcsname{\color{white}}%
      \expandafter\def\csname LTb\endcsname{\color{black}}%
      \expandafter\def\csname LTa\endcsname{\color{black}}%
      \expandafter\def\csname LT0\endcsname{\color{black}}%
      \expandafter\def\csname LT1\endcsname{\color{black}}%
      \expandafter\def\csname LT2\endcsname{\color{black}}%
      \expandafter\def\csname LT3\endcsname{\color{black}}%
      \expandafter\def\csname LT4\endcsname{\color{black}}%
      \expandafter\def\csname LT5\endcsname{\color{black}}%
      \expandafter\def\csname LT6\endcsname{\color{black}}%
      \expandafter\def\csname LT7\endcsname{\color{black}}%
      \expandafter\def\csname LT8\endcsname{\color{black}}%
    \fi
  \fi
    \setlength{\unitlength}{0.0500bp}%
    \ifx\gptboxheight\undefined%
      \newlength{\gptboxheight}%
      \newlength{\gptboxwidth}%
      \newsavebox{\gptboxtext}%
    \fi%
    \setlength{\fboxrule}{0.5pt}%
    \setlength{\fboxsep}{1pt}%
\begin{picture}(6802.00,4534.00)%
    \gplgaddtomacro\gplbacktext{%
      \csname LTb\endcsname%%
      \put(814,704){\makebox(0,0)[r]{\strut{}$0$}}%
      \put(814,1065){\makebox(0,0)[r]{\strut{}$50$}}%
      \put(814,1426){\makebox(0,0)[r]{\strut{}$100$}}%
      \put(814,1787){\makebox(0,0)[r]{\strut{}$150$}}%
      \put(814,2148){\makebox(0,0)[r]{\strut{}$200$}}%
      \put(814,2509){\makebox(0,0)[r]{\strut{}$250$}}%
      \put(814,2869){\makebox(0,0)[r]{\strut{}$300$}}%
      \put(814,3230){\makebox(0,0)[r]{\strut{}$350$}}%
      \put(814,3591){\makebox(0,0)[r]{\strut{}$400$}}%
      \put(814,3952){\makebox(0,0)[r]{\strut{}$450$}}%
      \put(814,4313){\makebox(0,0)[r]{\strut{}$500$}}%
      \put(946,484){\makebox(0,0){\strut{}$240$}}%
      \put(1856,484){\makebox(0,0){\strut{}$260$}}%
      \put(2766,484){\makebox(0,0){\strut{}$280$}}%
      \put(3676,484){\makebox(0,0){\strut{}$300$}}%
      \put(4585,484){\makebox(0,0){\strut{}$320$}}%
      \put(5495,484){\makebox(0,0){\strut{}$340$}}%
      \put(6405,484){\makebox(0,0){\strut{}$360$}}%
    }%
    \gplgaddtomacro\gplfronttext{%
      \csname LTb\endcsname%%
      \put(209,2508){\rotatebox{-270}{\makebox(0,0){\strut{}Frequency}}}%
      \put(3675,154){\makebox(0,0){\strut{}Number of visitors in a day}}%
    }%
    \gplbacktext
    \put(0,0){\includegraphics{plot2}}%
    \gplfronttext
  \end{picture}%
\endgroup

	\label{fig:plot2}
\end{figure}

\vspace{4mm}\noindent
\textit{Third data set is introduced}

\vspace{2mm}\noindent
In this third data set (plotted in figure \ref{fig:plot3}), the mean and median differ more.
Most numbers in this set is in the range of $[0,40]$ but there are many numbers outside of this span
as well.
In this case I think the median is the better representative of the data because the mean is skewed by
very far outliers.

Generally I think that the median is the better representative of a data set when there are a lot of
outlying numbers or when the variance is too big.  

\begin{figure}[h]
	\label{fig:plot3}
	\centering
	\caption{Histogram of experiment results}
	% GNUPLOT: LaTeX picture with Postscript
\begingroup
  \makeatletter
  \providecommand\color[2][]{%
    \GenericError{(gnuplot) \space\space\space\@spaces}{%
      Package color not loaded in conjunction with
      terminal option `colourtext'%
    }{See the gnuplot documentation for explanation.%
    }{Either use 'blacktext' in gnuplot or load the package
      color.sty in LaTeX.}%
    \renewcommand\color[2][]{}%
  }%
  \providecommand\includegraphics[2][]{%
    \GenericError{(gnuplot) \space\space\space\@spaces}{%
      Package graphicx or graphics not loaded%
    }{See the gnuplot documentation for explanation.%
    }{The gnuplot epslatex terminal needs graphicx.sty or graphics.sty.}%
    \renewcommand\includegraphics[2][]{}%
  }%
  \providecommand\rotatebox[2]{#2}%
  \@ifundefined{ifGPcolor}{%
    \newif\ifGPcolor
    \GPcolorfalse
  }{}%
  \@ifundefined{ifGPblacktext}{%
    \newif\ifGPblacktext
    \GPblacktexttrue
  }{}%
  % define a \g@addto@macro without @ in the name:
  \let\gplgaddtomacro\g@addto@macro
  % define empty templates for all commands taking text:
  \gdef\gplbacktext{}%
  \gdef\gplfronttext{}%
  \makeatother
  \ifGPblacktext
    % no textcolor at all
    \def\colorrgb#1{}%
    \def\colorgray#1{}%
  \else
    % gray or color?
    \ifGPcolor
      \def\colorrgb#1{\color[rgb]{#1}}%
      \def\colorgray#1{\color[gray]{#1}}%
      \expandafter\def\csname LTw\endcsname{\color{white}}%
      \expandafter\def\csname LTb\endcsname{\color{black}}%
      \expandafter\def\csname LTa\endcsname{\color{black}}%
      \expandafter\def\csname LT0\endcsname{\color[rgb]{1,0,0}}%
      \expandafter\def\csname LT1\endcsname{\color[rgb]{0,1,0}}%
      \expandafter\def\csname LT2\endcsname{\color[rgb]{0,0,1}}%
      \expandafter\def\csname LT3\endcsname{\color[rgb]{1,0,1}}%
      \expandafter\def\csname LT4\endcsname{\color[rgb]{0,1,1}}%
      \expandafter\def\csname LT5\endcsname{\color[rgb]{1,1,0}}%
      \expandafter\def\csname LT6\endcsname{\color[rgb]{0,0,0}}%
      \expandafter\def\csname LT7\endcsname{\color[rgb]{1,0.3,0}}%
      \expandafter\def\csname LT8\endcsname{\color[rgb]{0.5,0.5,0.5}}%
    \else
      % gray
      \def\colorrgb#1{\color{black}}%
      \def\colorgray#1{\color[gray]{#1}}%
      \expandafter\def\csname LTw\endcsname{\color{white}}%
      \expandafter\def\csname LTb\endcsname{\color{black}}%
      \expandafter\def\csname LTa\endcsname{\color{black}}%
      \expandafter\def\csname LT0\endcsname{\color{black}}%
      \expandafter\def\csname LT1\endcsname{\color{black}}%
      \expandafter\def\csname LT2\endcsname{\color{black}}%
      \expandafter\def\csname LT3\endcsname{\color{black}}%
      \expandafter\def\csname LT4\endcsname{\color{black}}%
      \expandafter\def\csname LT5\endcsname{\color{black}}%
      \expandafter\def\csname LT6\endcsname{\color{black}}%
      \expandafter\def\csname LT7\endcsname{\color{black}}%
      \expandafter\def\csname LT8\endcsname{\color{black}}%
    \fi
  \fi
    \setlength{\unitlength}{0.0500bp}%
    \ifx\gptboxheight\undefined%
      \newlength{\gptboxheight}%
      \newlength{\gptboxwidth}%
      \newsavebox{\gptboxtext}%
    \fi%
    \setlength{\fboxrule}{0.5pt}%
    \setlength{\fboxsep}{1pt}%
\begin{picture}(6802.00,5668.00)%
    \gplgaddtomacro\gplbacktext{%
      \csname LTb\endcsname%%
      \put(462,440){\makebox(0,0)[r]{\strut{}$0$}}%
      \put(462,996){\makebox(0,0)[r]{\strut{}$10$}}%
      \put(462,1553){\makebox(0,0)[r]{\strut{}$20$}}%
      \put(462,2109){\makebox(0,0)[r]{\strut{}$30$}}%
      \put(462,2665){\makebox(0,0)[r]{\strut{}$40$}}%
      \put(462,3222){\makebox(0,0)[r]{\strut{}$50$}}%
      \put(462,3778){\makebox(0,0)[r]{\strut{}$60$}}%
      \put(462,4334){\makebox(0,0)[r]{\strut{}$70$}}%
      \put(462,4891){\makebox(0,0)[r]{\strut{}$80$}}%
      \put(462,5447){\makebox(0,0)[r]{\strut{}$90$}}%
      \put(594,220){\makebox(0,0){\strut{}$0$}}%
      \put(1175,220){\makebox(0,0){\strut{}$20$}}%
      \put(1756,220){\makebox(0,0){\strut{}$40$}}%
      \put(2337,220){\makebox(0,0){\strut{}$60$}}%
      \put(2918,220){\makebox(0,0){\strut{}$80$}}%
      \put(3500,220){\makebox(0,0){\strut{}$100$}}%
      \put(4081,220){\makebox(0,0){\strut{}$120$}}%
      \put(4662,220){\makebox(0,0){\strut{}$140$}}%
      \put(5243,220){\makebox(0,0){\strut{}$160$}}%
      \put(5824,220){\makebox(0,0){\strut{}$180$}}%
      \put(6405,220){\makebox(0,0){\strut{}$200$}}%
    }%
    \gplgaddtomacro\gplfronttext{%
      \csname LTb\endcsname%%
      \put(5418,5274){\makebox(0,0)[r]{\strut{}Data frequency}}%
      \csname LTb\endcsname%%
      \put(5418,5054){\makebox(0,0)[r]{\strut{}Median}}%
      \csname LTb\endcsname%%
      \put(5418,4834){\makebox(0,0)[r]{\strut{}Mean}}%
    }%
    \gplbacktext
    \put(0,0){\includegraphics{plot3}}%
    \gplfronttext
  \end{picture}%
\endgroup

\end{figure}

\vspace{4mm}\noindent
\textit{How to calculate confidence interval for median value?}

\vspace{2mm}\noindent
To calculate the confidence interval for the median value you need the \textit{sample size}, \textit{quantile of interest}
(usually 0.5 for median) and \textit{z-value} (found in table). 

These values can be used to find the \textbf{indices} of the confidence interval, meaning the true median
lies in the range between the values at the indices (the data obviously has to be sorted for this to be true).

\section{Reproducibility in research}
I choose the following article: 

Kevin S. Killourhy and R. Maxion (2009). Comparing anomaly-detection
algorithms for keystroke dynamics.

\vspace{2mm}\noindent
I think they provide enough information to reproduce the experiment to some degree but some things will vary.

The authors do not provide the data they collected for further experimenting, they also don't provide the applications
used to gather the data, they instead just state that they developed an application to prompt users to input a password
and then the application recorded their input. They also state that they used an external reference clock to generate 
accurate timestamps for the user input but not further description of this clock or how it generated timestamps is provided.

The algorithm implementations are also not included in the article but references to where the algorithms are described'
in detail is provided.

I believe the data collection is the biggest obstacle in replicating the study, but I still think a similar experiment with
similar results is possible with what the authors have provided.


\end{document}