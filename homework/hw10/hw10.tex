\documentclass[11pt, a4paper]{article}

\usepackage{amsmath}
\usepackage{graphicx}

\title{Statistics and scientific reproducibility}
\author{Daniel Gustafsson}
\date{December 2021}

\begin{document}
\maketitle

\section{Confidence intervals and statistical significance}

I would probably use a confidence interval to present the accuracy 
of the result. A small confidence interval means that the my results
are statistically significant and that they can be trusted. 

I could also use a box plot to visualize my findings (like in figure \ref{fig:plot1}).
In figure \ref{fig:plot1} it is quite obvious that the measured values mostly agree
with each other and that the mean is $\sim 10$ 


\begin{figure}[h]
	\label{fig:plot1}
	\centering
	\caption{Box plot of experiment results}
	% GNUPLOT: LaTeX picture with Postscript
\begingroup
  \makeatletter
  \providecommand\color[2][]{%
    \GenericError{(gnuplot) \space\space\space\@spaces}{%
      Package color not loaded in conjunction with
      terminal option `colourtext'%
    }{See the gnuplot documentation for explanation.%
    }{Either use 'blacktext' in gnuplot or load the package
      color.sty in LaTeX.}%
    \renewcommand\color[2][]{}%
  }%
  \providecommand\includegraphics[2][]{%
    \GenericError{(gnuplot) \space\space\space\@spaces}{%
      Package graphicx or graphics not loaded%
    }{See the gnuplot documentation for explanation.%
    }{The gnuplot epslatex terminal needs graphicx.sty or graphics.sty.}%
    \renewcommand\includegraphics[2][]{}%
  }%
  \providecommand\rotatebox[2]{#2}%
  \@ifundefined{ifGPcolor}{%
    \newif\ifGPcolor
    \GPcolorfalse
  }{}%
  \@ifundefined{ifGPblacktext}{%
    \newif\ifGPblacktext
    \GPblacktexttrue
  }{}%
  % define a \g@addto@macro without @ in the name:
  \let\gplgaddtomacro\g@addto@macro
  % define empty templates for all commands taking text:
  \gdef\gplbacktext{}%
  \gdef\gplfronttext{}%
  \makeatother
  \ifGPblacktext
    % no textcolor at all
    \def\colorrgb#1{}%
    \def\colorgray#1{}%
  \else
    % gray or color?
    \ifGPcolor
      \def\colorrgb#1{\color[rgb]{#1}}%
      \def\colorgray#1{\color[gray]{#1}}%
      \expandafter\def\csname LTw\endcsname{\color{white}}%
      \expandafter\def\csname LTb\endcsname{\color{black}}%
      \expandafter\def\csname LTa\endcsname{\color{black}}%
      \expandafter\def\csname LT0\endcsname{\color[rgb]{1,0,0}}%
      \expandafter\def\csname LT1\endcsname{\color[rgb]{0,1,0}}%
      \expandafter\def\csname LT2\endcsname{\color[rgb]{0,0,1}}%
      \expandafter\def\csname LT3\endcsname{\color[rgb]{1,0,1}}%
      \expandafter\def\csname LT4\endcsname{\color[rgb]{0,1,1}}%
      \expandafter\def\csname LT5\endcsname{\color[rgb]{1,1,0}}%
      \expandafter\def\csname LT6\endcsname{\color[rgb]{0,0,0}}%
      \expandafter\def\csname LT7\endcsname{\color[rgb]{1,0.3,0}}%
      \expandafter\def\csname LT8\endcsname{\color[rgb]{0.5,0.5,0.5}}%
    \else
      % gray
      \def\colorrgb#1{\color{black}}%
      \def\colorgray#1{\color[gray]{#1}}%
      \expandafter\def\csname LTw\endcsname{\color{white}}%
      \expandafter\def\csname LTb\endcsname{\color{black}}%
      \expandafter\def\csname LTa\endcsname{\color{black}}%
      \expandafter\def\csname LT0\endcsname{\color{black}}%
      \expandafter\def\csname LT1\endcsname{\color{black}}%
      \expandafter\def\csname LT2\endcsname{\color{black}}%
      \expandafter\def\csname LT3\endcsname{\color{black}}%
      \expandafter\def\csname LT4\endcsname{\color{black}}%
      \expandafter\def\csname LT5\endcsname{\color{black}}%
      \expandafter\def\csname LT6\endcsname{\color{black}}%
      \expandafter\def\csname LT7\endcsname{\color{black}}%
      \expandafter\def\csname LT8\endcsname{\color{black}}%
    \fi
  \fi
    \setlength{\unitlength}{0.0500bp}%
    \ifx\gptboxheight\undefined%
      \newlength{\gptboxheight}%
      \newlength{\gptboxwidth}%
      \newsavebox{\gptboxtext}%
    \fi%
    \setlength{\fboxrule}{0.5pt}%
    \setlength{\fboxsep}{1pt}%
\begin{picture}(6802.00,4534.00)%
    \gplgaddtomacro\gplbacktext{%
      \csname LTb\endcsname%%
      \put(814,704){\makebox(0,0)[r]{\strut{}$1$}}%
      \put(814,2509){\makebox(0,0)[r]{\strut{}$10$}}%
      \put(814,4313){\makebox(0,0)[r]{\strut{}$100$}}%
      \put(946,484){\makebox(0,0){\strut{}$1$}}%
      \put(1996,484){\makebox(0,0){\strut{}$32$}}%
      \put(3046,484){\makebox(0,0){\strut{}$1024$}}%
      \put(4095,484){\makebox(0,0){\strut{}$32768$}}%
      \put(5145,484){\makebox(0,0){\strut{}$1.04858\times10^{6}$}}%
      \put(6195,484){\makebox(0,0){\strut{}$3.35544\times10^{7}$}}%
    }%
    \gplgaddtomacro\gplfronttext{%
      \csname LTb\endcsname%%
      \put(209,2508){\rotatebox{-270}{\makebox(0,0){\strut{}$log$(Average payoff)}}}%
      \put(3675,154){\makebox(0,0){\strut{}$log_2$(Number of simulations)}}%
      \csname LTb\endcsname%%
      \put(5418,4140){\makebox(0,0)[r]{\strut{}Data}}%
    }%
    \gplbacktext
    \put(0,0){\includegraphics{plot1}}%
    \gplfronttext
  \end{picture}%
\endgroup

\end{figure}

\vspace{4mm}\noindent
Figure \ref{fig:plot2} presents the difference in the data sets. Just from looking at the
figure I would claim the results are consistent with each other.

\[\text{Mean original} = 9.8689, \text{ Mean new} = 9.63796\]
\[\text{Median original} = 9.921014, \text{ Median new} = 9.757105\]


\begin{figure}[h]
	\label{fig:plot2}
	\centering
	\caption{Box plots of the different data sets}
	% GNUPLOT: LaTeX picture with Postscript
\begingroup
  \makeatletter
  \providecommand\color[2][]{%
    \GenericError{(gnuplot) \space\space\space\@spaces}{%
      Package color not loaded in conjunction with
      terminal option `colourtext'%
    }{See the gnuplot documentation for explanation.%
    }{Either use 'blacktext' in gnuplot or load the package
      color.sty in LaTeX.}%
    \renewcommand\color[2][]{}%
  }%
  \providecommand\includegraphics[2][]{%
    \GenericError{(gnuplot) \space\space\space\@spaces}{%
      Package graphicx or graphics not loaded%
    }{See the gnuplot documentation for explanation.%
    }{The gnuplot epslatex terminal needs graphicx.sty or graphics.sty.}%
    \renewcommand\includegraphics[2][]{}%
  }%
  \providecommand\rotatebox[2]{#2}%
  \@ifundefined{ifGPcolor}{%
    \newif\ifGPcolor
    \GPcolorfalse
  }{}%
  \@ifundefined{ifGPblacktext}{%
    \newif\ifGPblacktext
    \GPblacktexttrue
  }{}%
  % define a \g@addto@macro without @ in the name:
  \let\gplgaddtomacro\g@addto@macro
  % define empty templates for all commands taking text:
  \gdef\gplbacktext{}%
  \gdef\gplfronttext{}%
  \makeatother
  \ifGPblacktext
    % no textcolor at all
    \def\colorrgb#1{}%
    \def\colorgray#1{}%
  \else
    % gray or color?
    \ifGPcolor
      \def\colorrgb#1{\color[rgb]{#1}}%
      \def\colorgray#1{\color[gray]{#1}}%
      \expandafter\def\csname LTw\endcsname{\color{white}}%
      \expandafter\def\csname LTb\endcsname{\color{black}}%
      \expandafter\def\csname LTa\endcsname{\color{black}}%
      \expandafter\def\csname LT0\endcsname{\color[rgb]{1,0,0}}%
      \expandafter\def\csname LT1\endcsname{\color[rgb]{0,1,0}}%
      \expandafter\def\csname LT2\endcsname{\color[rgb]{0,0,1}}%
      \expandafter\def\csname LT3\endcsname{\color[rgb]{1,0,1}}%
      \expandafter\def\csname LT4\endcsname{\color[rgb]{0,1,1}}%
      \expandafter\def\csname LT5\endcsname{\color[rgb]{1,1,0}}%
      \expandafter\def\csname LT6\endcsname{\color[rgb]{0,0,0}}%
      \expandafter\def\csname LT7\endcsname{\color[rgb]{1,0.3,0}}%
      \expandafter\def\csname LT8\endcsname{\color[rgb]{0.5,0.5,0.5}}%
    \else
      % gray
      \def\colorrgb#1{\color{black}}%
      \def\colorgray#1{\color[gray]{#1}}%
      \expandafter\def\csname LTw\endcsname{\color{white}}%
      \expandafter\def\csname LTb\endcsname{\color{black}}%
      \expandafter\def\csname LTa\endcsname{\color{black}}%
      \expandafter\def\csname LT0\endcsname{\color{black}}%
      \expandafter\def\csname LT1\endcsname{\color{black}}%
      \expandafter\def\csname LT2\endcsname{\color{black}}%
      \expandafter\def\csname LT3\endcsname{\color{black}}%
      \expandafter\def\csname LT4\endcsname{\color{black}}%
      \expandafter\def\csname LT5\endcsname{\color{black}}%
      \expandafter\def\csname LT6\endcsname{\color{black}}%
      \expandafter\def\csname LT7\endcsname{\color{black}}%
      \expandafter\def\csname LT8\endcsname{\color{black}}%
    \fi
  \fi
    \setlength{\unitlength}{0.0500bp}%
    \ifx\gptboxheight\undefined%
      \newlength{\gptboxheight}%
      \newlength{\gptboxwidth}%
      \newsavebox{\gptboxtext}%
    \fi%
    \setlength{\fboxrule}{0.5pt}%
    \setlength{\fboxsep}{1pt}%
\begin{picture}(6802.00,4534.00)%
    \gplgaddtomacro\gplbacktext{%
      \csname LTb\endcsname%%
      \put(814,704){\makebox(0,0)[r]{\strut{}$0$}}%
      \put(814,1065){\makebox(0,0)[r]{\strut{}$50$}}%
      \put(814,1426){\makebox(0,0)[r]{\strut{}$100$}}%
      \put(814,1787){\makebox(0,0)[r]{\strut{}$150$}}%
      \put(814,2148){\makebox(0,0)[r]{\strut{}$200$}}%
      \put(814,2509){\makebox(0,0)[r]{\strut{}$250$}}%
      \put(814,2869){\makebox(0,0)[r]{\strut{}$300$}}%
      \put(814,3230){\makebox(0,0)[r]{\strut{}$350$}}%
      \put(814,3591){\makebox(0,0)[r]{\strut{}$400$}}%
      \put(814,3952){\makebox(0,0)[r]{\strut{}$450$}}%
      \put(814,4313){\makebox(0,0)[r]{\strut{}$500$}}%
      \put(946,484){\makebox(0,0){\strut{}$240$}}%
      \put(1856,484){\makebox(0,0){\strut{}$260$}}%
      \put(2766,484){\makebox(0,0){\strut{}$280$}}%
      \put(3676,484){\makebox(0,0){\strut{}$300$}}%
      \put(4585,484){\makebox(0,0){\strut{}$320$}}%
      \put(5495,484){\makebox(0,0){\strut{}$340$}}%
      \put(6405,484){\makebox(0,0){\strut{}$360$}}%
    }%
    \gplgaddtomacro\gplfronttext{%
      \csname LTb\endcsname%%
      \put(209,2508){\rotatebox{-270}{\makebox(0,0){\strut{}Frequency}}}%
      \put(3675,154){\makebox(0,0){\strut{}Number of visitors in a day}}%
    }%
    \gplbacktext
    \put(0,0){\includegraphics{plot2}}%
    \gplfronttext
  \end{picture}%
\endgroup

\end{figure}

\end{document}