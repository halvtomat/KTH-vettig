\documentclass[11pt,a4paper]{article}

\usepackage{amsmath}

\title{Panel discussions}
\author{Daniel Gustafsson}
\date{October 2021}

\begin{document}
\maketitle

\section{Is Computer Science Science?}
\subsection{Introduction}
Is computer science science? Not everyone agrees on the answer, some argue that since computers are man-made objects they can not have
their own science, the science of computers is just mathematics and physics. Others argue that computers have gone far beyond the basic
mathematics and physics of the past and it is now justifiable to let computers have their own science area. All agree that experts in
field of computers are great engineers and even artist but are they scientists?

\subsection{Moderator questions}
At what point in time did/will computer science become its own area of science and stopped/stop being physics and mathematics?

What area of computer science would be qualified as a science? Building a website? Experimenting with a new network protocol? A simple calculator application?

\subsection{Arguments}
\textit{Computer science is just applied mathematics and not a science in it self}

While this is true, the same thing can be said about all (maybe?) sciences, physics is applied mathematics,
chemistry is applied physics which is mathematics etc.

\textit{Most computer scientists use already existing solutions to solve their problems, instead of creating a sorting function you write .sort() etc}

This is true but every problem is still unique and therefore requires a unique solution. Most physicists also just use the existing formulas created centuries
ago to solve their new unique problems. This doesn't make it less of a science.

\textit{Computer science have only existed for a few decades and most areas are already explored and depleted, it is just a phase and not a real science}

Yes computer science is very new to this world but it never stopped expanding. Current research in AI, image recognition, block-chain or
virtual reality technology are great examples of modern computer science areas which are far from being depleted. Computer science has proven
again and again that it is here to stay.  

\section{Can AI algorithms be allowed to make sensitive decisions involving humans?}
\subsection{A case of algorithmic bias}
One case of algorithmic bias I found was that of facial recognition algorithms struggling to identify certain demographics.
In a study of some common both commercial and non-commercial facial recognition algorithms, it was found that a person who
is part of either groups women, dark skinned or young adult (18 - 30 years old) would have a lower probability of being
recognized by the algorithms \cite{ai-bias}. An 25 year old African woman might only have 65\% probability of being recognized
while a 40 year old European man would have 97\%.   

\subsection{Introduction}
Artificial intelligence is something most people might consider science fiction but it is a great part of society even today.
Several governments and companies around the world uses AI to assist decision making in areas such as health care, banking or policing.
The problem with this is that the AI's are not yet perfect and some times discriminate against certain groups or demographics.
Can AI be allowed to make sensitive decisions involving humans?

\subsection{Moderator questions}
Who is at fault when an AI makes a faulty decision that leads to human consequences?

To what degree is a software engineer responsible for the decisions of the AI he/she designed?

\subsection{Arguments}
I am for AI decision making.

\textit{AIs uses logic and probability to decide, humans are always biased to some degree therefore AIs are always less biased than humans}

An AI is but a reflection of the humans that created it and keeps the biases those humans have/had.

\textit{As long as the AI is extensively tested we have a reason to trust it with sensitive decisions}

It is impossible to test all corner cases.

\textit{Whereas a human might have a bad day because of some unforeseen circumstance, an AI will never have a bad day and will always perform at its best}

The AIs servers could crash and it would have a bad day.

\bibliographystyle{amsplain}
\bibliography{hw4.bib}

\end{document}