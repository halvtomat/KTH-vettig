\documentclass[11pt,a4paper]{article}

\begin{document}
\title{A Paradox, Induction, and Reserch Basics}
\author{Daniel Gustafsson}
\date{September 2021}
\maketitle

\section{Achilles and the Tortoise}
The paradox in \textit{Achilles and the tortoise} lies in the mathematical formulation of the problem.
Zeno uses an infinite series of distances decreasing in size to define a larger distance and then argues that to travel the larger distance,
one would have to first travel the previous smaller distance and to do that one would have to travel the previous smaller distance and so on.
He then argues that traveling an infinite set of distances is impossible and therefore it is impossible to travel the larger distance.
I think that the only solution to this paradox is to define a shortest possible unit of distance which cant be divided. 
If such a distance exists the series is no longer infinite and the paradox is no longer a paradox.


\section{When does induction work?}



\section{Reserch basics}
	
\end{document}