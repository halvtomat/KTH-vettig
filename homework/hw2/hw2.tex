\documentclass[11pt,a4paper]{article}

\usepackage{amsmath}

\begin{document}
\title{A Paradox, Induction, and Research Basics}
\author{Daniel Gustafsson}
\date{September 2021}
\maketitle

\section{Achilles and the Tortoise}
The paradox in \textit{Achilles and the tortoise} lies in the mathematical formulation of the problem.
Zeno uses an infinite series of distances decreasing in size to define a larger distance and then argues that to travel the larger distance,
one would have to first travel the previous smaller distance and to do that one would have to travel the previous smaller distance and so on.
He then argues that traveling an infinite set of distances is impossible and therefore it is impossible to travel the larger distance.
I think that the only solution to this paradox is to define a shortest possible unit of distance which cant be divided. 
If such a distance exists the series is no longer infinite and the paradox is no longer a paradox.


\section{When does induction work?}
The \textit{Raven Paradox} claims \textit{all ravens are black} and that observing something \textit{not black} that is also \textit{not a raven}
provides proof in favor of this claim.

In my opinion the paradox is not a paradox, I agree that observing an object that is \textit{not black and not a raven} provides a tiny bit of proof
that \textit{all ravens are black}. My reasoning for this is that there is a finite set of observable objects, some are black ravens and some are
non-black non-ravens. If a non-black raven exists it must be in the set of observable objects. Every time an observation is made of a non-black non-raven
there is one less object in the set of observable objects that could potentially be a non-black raven.

Two identical arrays were sorted, one by quicksort and one by insertion sort. Insertion sort was faster in this scenario,
therefore insertion sort is the faster then quicksort. This statement is obviously false because quicksort is faster in most cases.
There could have been any number of similar tests with the same results if the arrays in the tests were always either short or close to being
sorted already. In this case we are drawing conclusions from past observations much like in Goodman's paradox.

\section{Research basics}
In a modern research project on artificial intelligence in games, the authors \textit{compares} AI in computer games to the graphics and physics
of computer games and \textit{evaluates} that AI is or can be an equally important selling point as graphics or physics has been\cite{ai_games}. 
The authors also states that AI has become the technical core of improving the playability of a game and they \textit{predict} that AI development
will have a great impact on the future of gaming.

Another interesting study, on current blockchain technology\cite{eth}, \textit{compares} the operating cost of currently used relay schemes and a new
validate-on-demand pattern that they call ETH Relay. Their conclusions are that their new relay scheme could reduce operating costs by as much as 92\%

Computer science research is mostly conducted with a positivist approach. When inventing a new algorithm to solve some problem, a computer scientist
will try to prove the validity of the algorithm using mathematical models and quantitative analysis. Examples of this exists everywhere in computer science.
The way we describe an algorithms time or memory complexity with big O notation is purely mathematical, another example is cryptography which is the base
of all encryption and cyber security, cryptography is also purely mathematical.

\bibliographystyle{amsplain}
\bibliography{hw2.bib}

\end{document}