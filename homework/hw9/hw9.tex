\documentclass[11pt, a4paper]{article}

\usepackage{amsmath}
\usepackage{graphicx}
\usepackage{listings}

\title{Probability and simulation}
\author{Daniel Gustafsson}
\date{November 2021}

\begin{document}
\maketitle

\section{St. Petersburg paradox}

\textit{What would you yourself be willing to pay to enter the game?}

\vspace{2mm}\noindent
It depends on the the \textit{maximum amount} I would be able to win.
If the amount is for example \textit{1 million} SEK, then I would be willing to pay 20 SEK to enter the game.
I calculated this based on the utility of the game with the following function where x is the resulting
number of SEK I would be willing to bet.

\[x = log_2(\text{maximum amount})\]

This of course is different if the \textit{maximum amount} is infinite, I would not be willing to pay an infinite
amount to play the game, I would probably have an upper limit of 370 SEK ($\sim$ 40 USD) just because if I won that game I would
be at the top of the Forbes list of richest people and I don't think I would ever need more money than that.

\textit{The formulation of the game and the estimate of its value as infinity makes several unrealistic assumptions,
both from the player's point of view and from the casino's point of view. What are these assumptions?}

\vspace{2mm}\noindent
The first and most obvious assumption is that the casino can't pay an infinite amount.

\vspace{1mm}\noindent
The second assumption is that the utility of each ducat is the same. Obviously I (the player) don't have
an infinite amount of money before the game and adding some ridiculous amount to the already ridiculous highest winning amount
wouldn't persuade me to risk more of my money now. 

The utility of winning 1 trillion ducats and 3 trillion ducats would be identical but the negative utility of losing x ducats is less
than the negative utility of losing x + 1 ducats.

\begin{figure}[ht]
	\label{fig:st_petersburg}
	\centering
	\caption{Logarithmic plot of average payoff over the last $n$ games as a function of $n$. 28 simulations with $2^x$ games in each 
	where x is increasing by 1 each simulation.
	Maximum winnings limited to 10 million.}
	\scriptsize
	\input{plot1}
\end{figure}

\textit{Plot discussion}

\vspace{2mm}\noindent
The line in \ref{fig:st_petersburg} is obviously converging, I initially though that it was converging towards the expected value, in this case $\sim 23$,
but it seems to be converging towards half of that value instead. I don't know why this happens.

\lstinputlisting[label={lst:sim1}, caption={Simulation 1 code}, language=Python, firstline=9, lastline=39, tabsize=4]{simulation1.py}
\section{Hospital visitors}

As seen in \ref{fig:hospital}, from 10 simulated years (3650 days), the frequency of days when visitors where 360 or above is practically none. This leads me to suspect that the 
crowded day with 369 visitors was related to some other cause(s).

\begin{figure}[ht]
	\label{fig:hospital}
	\centering
	\caption{Histogram of visitors per day in the hospital simulation}
	\scriptsize
	% GNUPLOT: LaTeX picture with Postscript
\begingroup
  \makeatletter
  \providecommand\color[2][]{%
    \GenericError{(gnuplot) \space\space\space\@spaces}{%
      Package color not loaded in conjunction with
      terminal option `colourtext'%
    }{See the gnuplot documentation for explanation.%
    }{Either use 'blacktext' in gnuplot or load the package
      color.sty in LaTeX.}%
    \renewcommand\color[2][]{}%
  }%
  \providecommand\includegraphics[2][]{%
    \GenericError{(gnuplot) \space\space\space\@spaces}{%
      Package graphicx or graphics not loaded%
    }{See the gnuplot documentation for explanation.%
    }{The gnuplot epslatex terminal needs graphicx.sty or graphics.sty.}%
    \renewcommand\includegraphics[2][]{}%
  }%
  \providecommand\rotatebox[2]{#2}%
  \@ifundefined{ifGPcolor}{%
    \newif\ifGPcolor
    \GPcolorfalse
  }{}%
  \@ifundefined{ifGPblacktext}{%
    \newif\ifGPblacktext
    \GPblacktexttrue
  }{}%
  % define a \g@addto@macro without @ in the name:
  \let\gplgaddtomacro\g@addto@macro
  % define empty templates for all commands taking text:
  \gdef\gplbacktext{}%
  \gdef\gplfronttext{}%
  \makeatother
  \ifGPblacktext
    % no textcolor at all
    \def\colorrgb#1{}%
    \def\colorgray#1{}%
  \else
    % gray or color?
    \ifGPcolor
      \def\colorrgb#1{\color[rgb]{#1}}%
      \def\colorgray#1{\color[gray]{#1}}%
      \expandafter\def\csname LTw\endcsname{\color{white}}%
      \expandafter\def\csname LTb\endcsname{\color{black}}%
      \expandafter\def\csname LTa\endcsname{\color{black}}%
      \expandafter\def\csname LT0\endcsname{\color[rgb]{1,0,0}}%
      \expandafter\def\csname LT1\endcsname{\color[rgb]{0,1,0}}%
      \expandafter\def\csname LT2\endcsname{\color[rgb]{0,0,1}}%
      \expandafter\def\csname LT3\endcsname{\color[rgb]{1,0,1}}%
      \expandafter\def\csname LT4\endcsname{\color[rgb]{0,1,1}}%
      \expandafter\def\csname LT5\endcsname{\color[rgb]{1,1,0}}%
      \expandafter\def\csname LT6\endcsname{\color[rgb]{0,0,0}}%
      \expandafter\def\csname LT7\endcsname{\color[rgb]{1,0.3,0}}%
      \expandafter\def\csname LT8\endcsname{\color[rgb]{0.5,0.5,0.5}}%
    \else
      % gray
      \def\colorrgb#1{\color{black}}%
      \def\colorgray#1{\color[gray]{#1}}%
      \expandafter\def\csname LTw\endcsname{\color{white}}%
      \expandafter\def\csname LTb\endcsname{\color{black}}%
      \expandafter\def\csname LTa\endcsname{\color{black}}%
      \expandafter\def\csname LT0\endcsname{\color{black}}%
      \expandafter\def\csname LT1\endcsname{\color{black}}%
      \expandafter\def\csname LT2\endcsname{\color{black}}%
      \expandafter\def\csname LT3\endcsname{\color{black}}%
      \expandafter\def\csname LT4\endcsname{\color{black}}%
      \expandafter\def\csname LT5\endcsname{\color{black}}%
      \expandafter\def\csname LT6\endcsname{\color{black}}%
      \expandafter\def\csname LT7\endcsname{\color{black}}%
      \expandafter\def\csname LT8\endcsname{\color{black}}%
    \fi
  \fi
    \setlength{\unitlength}{0.0500bp}%
    \ifx\gptboxheight\undefined%
      \newlength{\gptboxheight}%
      \newlength{\gptboxwidth}%
      \newsavebox{\gptboxtext}%
    \fi%
    \setlength{\fboxrule}{0.5pt}%
    \setlength{\fboxsep}{1pt}%
\begin{picture}(6802.00,4534.00)%
    \gplgaddtomacro\gplbacktext{%
      \csname LTb\endcsname%%
      \put(814,704){\makebox(0,0)[r]{\strut{}$0$}}%
      \put(814,1065){\makebox(0,0)[r]{\strut{}$50$}}%
      \put(814,1426){\makebox(0,0)[r]{\strut{}$100$}}%
      \put(814,1787){\makebox(0,0)[r]{\strut{}$150$}}%
      \put(814,2148){\makebox(0,0)[r]{\strut{}$200$}}%
      \put(814,2509){\makebox(0,0)[r]{\strut{}$250$}}%
      \put(814,2869){\makebox(0,0)[r]{\strut{}$300$}}%
      \put(814,3230){\makebox(0,0)[r]{\strut{}$350$}}%
      \put(814,3591){\makebox(0,0)[r]{\strut{}$400$}}%
      \put(814,3952){\makebox(0,0)[r]{\strut{}$450$}}%
      \put(814,4313){\makebox(0,0)[r]{\strut{}$500$}}%
      \put(946,484){\makebox(0,0){\strut{}$240$}}%
      \put(1856,484){\makebox(0,0){\strut{}$260$}}%
      \put(2766,484){\makebox(0,0){\strut{}$280$}}%
      \put(3676,484){\makebox(0,0){\strut{}$300$}}%
      \put(4585,484){\makebox(0,0){\strut{}$320$}}%
      \put(5495,484){\makebox(0,0){\strut{}$340$}}%
      \put(6405,484){\makebox(0,0){\strut{}$360$}}%
    }%
    \gplgaddtomacro\gplfronttext{%
      \csname LTb\endcsname%%
      \put(209,2508){\rotatebox{-270}{\makebox(0,0){\strut{}Frequency}}}%
      \put(3675,154){\makebox(0,0){\strut{}Number of visitors in a day}}%
    }%
    \gplbacktext
    \put(0,0){\includegraphics{plot2}}%
    \gplfronttext
  \end{picture}%
\endgroup

\end{figure}

\lstinputlisting[label={lst:sim2}, caption={Simulation 2 code}, firstline=7, lastline=22, tabsize=4, language=Python]{simulation2.py}


\end{document}