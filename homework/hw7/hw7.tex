\documentclass[11pt, a4paper]{article}

\usepackage{amsmath}

\title{Planning an experiment}
\author{Daniel Gustafsson}
\date{November 2021}

\begin{document}
\maketitle

\section{Experimental question}
Can modern artificial intelligence and machine learning techniques be used to
simulate human behavior better than traditional methods in computer controlled characters in games?

\vspace{2mm}\noindent According to \cite{ai_in_games}, AI has become one of the main selling points in
games but it also one of the biggest technical challenges.
Historically, game developers use a state machine to decide how a non-player character should behave.
Simulating human behavior this way is often overly complex and a player can easily understand that
a non-player character is simulated and predict its behavior.

Artificial intelligence and machine learning technology have had great advances in recent years,
but has it advanced far enough to be successfully used to simulate human behavior in games?

\section{How to measure?}
In \cite{turing_test}, a Turing test for "Bots" (non-player characters) is suggested.
A game is played with both human and computer players where no game character can be identified
by anything other than behavior and a randomly generated name.
After this game, a set of judges rate each player using a scale of 1-5 where 1 is 
"This player is a not very human-like bot" and 5 is "This player is human".

This test could be used to experiment with different behavior algorithms for non-player characters.

\section{The experiment}
Use some different AI and ML techniques like reinforcement learning or neural nets to
program behavior for game bots. Start a game where the game characters are controlled by a mix of
humans, traditional state machine bots and a set of different bots using AI or ML.
Let the game play out and finally rate each game character on whether it is human or computer controlled.

The traditional bots will set the baseline rating. If a more advanced bot (using AI or ML technology) is
consistently given a higher score, a conclusion can be reached that AI and ML technology can be used
to simulate human behavior better than traditional methods. If instead all advanced bots are given similar
or lower scores than the traditional bots, the conclusion that traditional methods are still superior in 
mimicking human behavior can be drawn.

For this experiment to be made possible, teams of specialists in non-player character behavior and AI/ML
is required. Additionally a number of human subjects are required as both players and judges.

\section{Objections to experiments}
One of the objections in Walter Tichy's article \cite{walter} is that demonstration will suffice, I think
this objection could be applied to my suggested experiment. The time and effort of the experts and specialists
required in the experiment would definitively be significant, it might be easier to just ask a few developers
to get creative with a few different techniques and then demo their results.

In contrast to this, I think that simulating human behavior in a non-player character is a very complex subject
and that a few creative developers wouldn't do the experiment justice. To get an accurate result, I think an
experiment is required.

\bibliographystyle{amsplain}
\bibliography{hw7.bib}

\end{document}